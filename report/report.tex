\documentclass[15pt]{article}

\title{Infinite Precision Calculator\\ \large Final Project Report for CS1023 - Software Development Fundamentals}
\author{Devansh Tripathi - CS24BTECH11022}
\date{\today}


\usepackage{fancyhdr}
\pagestyle{fancy}
\fancyhf{}
\rhead{CS1023 Final Project}
\lhead{Devansh Tripathi}
\rfoot{\thepage}

\begin{document}
\maketitle

\newpage
\section{Abstract}

This Java library contains classes and methods to enable infinite precision arithmetic 
operations for integers and float types. This solves the problem of doing addition, 
subtraction, multiplication and division on large numbers that may overflow the bounds 
of a regular \texttt{int} or \texttt{float}, or even \texttt{long} or \texttt{double}, or it may be the case with 
their result (eg: multiplying two very big numbers or infinite division).\\

\textbf{The main challenges in the project were:}\\
- Performing the operations on numbers with different number of digits.\\
- Operations on float types as they have integral and fractional parts.\\
- Logic for handling position of decimal points in the case of multiplication or division of float types and acheiving precision upto n places.\\
- Dealing with negative numbers and handling cases for each operation.

\end{document}