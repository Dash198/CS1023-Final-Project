\documentclass[15pt]{article}

\title{Infinite Precision Calculator\\ \large Final Project Report for CS1023 - Software Development Fundamentals}
\author{Devansh Tripathi - CS24BTECH11022}
\date{\today}


\usepackage{fancyhdr}
\pagestyle{fancy}
\fancyhf{}
\rhead{CS1023 Final Project}
\lhead{Devansh Tripathi}
\rfoot{\thepage}

\begin{document}
\begin{titlepage}
    \centering
    \vspace*{2cm}

    {\huge \bfseries Infinite Precision Calculator \par}
    \vspace{0.5cm}

    {\Large Final Project Report for CS1023\\Software Development Fundamentals \par}
    \vspace{1.5cm}

    {\Large \textbf{Submitted by:} \par} 
    \vspace{0.3cm}
    {\large Devansh Tripathi \par} 
    {\large CS24BTECH11022 \par} 
    \vspace{2cm}

    {\large \textbf{Date:} \today \par} 

    \vfill
    {\large Indian Institute of Technology Hyderabad \par}
\end{titlepage}

\newpage
\section{Abstract}

This Java library contains classes and methods to enable infinite precision arithmetic 
operations for integers and float types. This solves the problem of doing addition, 
subtraction, multiplication and division on large numbers that may overflow the bounds 
of a regular \texttt{int} or \texttt{float}, or even \texttt{long} or \texttt{double}, or it may be the case with 
their result (eg: multiplying two very big numbers or infinite division).\\

\textbf{The main challenges in the project were:}
\begin{itemize}
    \item Performing the operations on numbers with different number of digits.
    \item Operations on float types as they have integral and fractional parts.
    \item Logic for handling position of decimal points in the case of multiplication or division of float types and acheiving precision upto n places.
    \item Dealing with negative numbers and handling cases for each operation.
\end{itemize}


\section{Introduction}

Precision is quite important for complex calculations where even small changes can 
drastically change the final result. Hence, this library tries to present a solution 
to this problem of bounded values of standard \texttt{int}, \texttt{long}, \texttt{float} and \texttt{double} types in 
Java by storing the numbers as Strings, and then doing the required operation digit-wise 
and storing the result into a String as well. Since there is no explicit bound on the 
size of a String in Java, this solves the problem of precision as the result can be as 
long as desired.\\

\textbf{Data range of the standard data types in Java:}
\begin{itemize}
    \item  int : -2147483648 to 2147483647
    \item long : -9223372036854775808 to 9223372036854775807
    \item float : 1.40239846e-45 to 3.40282347e+38
    \item double : 4.94065645841246544e-324 to 1.79769313486231570e+308
\end{itemize}

\textbf{What this library supports:}
\begin{itemize}
    \item Addition, Subtraction, Multiplication and Divison of numbers of arbitrary length, even those exceeding the data range of the standard data types.
    \item Division producing a result capable of having an arbitrary amount of decimal places, it has been set to 1000 but can be extended.
\end{itemize}

\newpage

\section{Features}

\textbf{A brief description of the main components of this library:}
\begin{itemize}
    \item \texttt{AInteger}
    \begin{itemize}
        \item This class handles parsing integers(small and large) and all the integer-related operations.
        \item Each \texttt{AInteger} object stores the number given to it as a String.
        \item Since the numbers are stored as Strings, the length of the numbers can be arbitrary.
        \item Supports addition, subtraction, multiplication, and division.
    \end{itemize}
    
    \item \texttt{AFloat}
    \begin{itemize}
        \item This class handles all the float-related operations and parsing.
        \item Each \texttt{AFloat} object stores the floating point number given to it as a String.
        \item The number can be arbitrary precision- there is no bound on it from either side.
        \item Supports addition, subtraction, multiplication and division upto arbitrary precision.
    \end{itemize}
    
    \item \texttt{MyInfArith}
    \begin{itemize}
        \item Although this class is not a part of the package, it is still a convenient way to use the above classes via the command line.
        \item Usage of this class is given in the \textbf{Build and Run} section.
    \end{itemize}
\end{itemize}

\section{Implementation}

\begin{itemize}
    \item Language: \textbf{Java 21}
    
    \item Build Tool used: \textbf{Maven}
    
    \item Structure: The main source files are located in the \texttt{src/main/java/arbitraryarithmetic}
    directory, namely \texttt{AInteger.java} and \texttt{AFloat.java}.

    \item Each number is stored in an object which is either \texttt{AInteger} or 
    \texttt{AFloat}. The number is stored as a private variable \texttt{value}, which
    can be accessed by using the accessor function \texttt{getValue()}.

    \item Operations are implemented as methods, so the four operations are implemented as 
    \texttt{add}, \texttt{subtract}, \texttt{multiply} and \texttt{divide} in either class.

    \item To perform an operation between \texttt{x1} and \texttt{x2}, use the statment 
    \texttt{x1.operation\_name(x2)}, for example if we want to add two \texttt{AInteger} objects \texttt{a} and \texttt{b}, 
    we would use \texttt{a.add(b)}.

    \item Performing an operation will return the result of the operation as an object 
    of the same type. The value of the result can be accessed by the \texttt{getValue()} function.

    \item The main challenges encountered in the implementation were:
    
    \begin{itemize}
        \item Performing the operations on numbers with different number of digits.
        \item Operations on float types as they have integral and fractional parts.
        \item Logic for handling position of decimal points in the case of multiplication or division of float types and acheiving precision upto $n$ places.
        \item Dealing with negative numbers and handling cases for each operation.
    
    \end{itemize}
\end{itemize}

\section{Testing}

    Correctness of the functions of these classes were mainly testes manually, and all the
    cases used can be found in \texttt{raw\_source/test\_files/} directory.

    \textbf{Here are some example cases:}

    \subsection{AInteger}

    \begin{enumerate}
        \item 123 + 456 = 579
        \item -123 + 456 = 333
        \item 123 + -456 = -333
        \item -123 + -456 = -579
        \item 1000000000000000000000000 + 1 = 1000000000000000000000001
        \item 999999999999999999999999 + 1 = 1000000000000000000000000
        \item 1000000000000000000000000 - 999999999999999999999999 = 1
        \item -999999999999999999999999 + -1 = -1000000000000000000000000
        \item 0 + 0 = 0
        \item 0 - 0 = 0
        \item 0 - 123 = -123
        \item 123 - 0 = 123
    \end{enumerate}

    \subsection{AFloat}

    \begin{enumerate}
        \item 1.0 / 3.0 = 0.333333.... (1000 places)
        \item 2.0 / 3.0 = 0.666666.... (1000 places)
        \item 0.0000000000001 + 0.0000000000001 = 0.0000000000002
        \item 999999999999.999999999999 + 0.000000000001 = 1000000000000.0
        \item 999999999999.999 * 1.001 = 1000999999999.998999
        \item -123456.789 * -1.0 = 123456.789
        \item 1.000000 / 1000000.0 = 0.000001
        \item -0.0 + 0.0 = 0.0
        \item -0.0 - -0.0 = 0.0
        \item 0.0 * 0.1 = 0.0
        \item 0.0 / 0.1 = 0.0
        \item 1.0000000000001 - 1.0000000000000 = 0.0000000000001
        \item 1000.0000001 - 999.9999999 = 0.0000002
    \end{enumerate}


    All these answers can be verified to be correct.

\section{Build and Run} \label{bnr}

\begin{itemize}
    \item Dependecies: \textbf{Java 21}, \textbf{Maven}
    \item How to build the package (.JAR):
    
    Run the following command in the terminal from the project directory:\\
    \texttt{mvn clean package}

    \item How to link the package to your file:
    
    \begin{itemize}
        \item First, compile your file using the command:\\
        \texttt{javac -cp .:<add the absolute path to the aarithmetic.jar package here> <relative path to your desired file>}

        \item Next, run your file using:\\
        \texttt{java -cp .:<absolute path to aarithmetic.jar> <relative path to your file>}
    \end{itemize}

    \item If you just want to test out the package functions or perform simple operations, there are two ways to do this:
    \
    \begin{itemize}
        \item Link the package to \texttt{MyInfArith/MyInfArith.java} and then run it with the following command:\\
        \texttt{java -cp .:<absolute path to aarithmetic.jar> MyInfArith.MyInfArith <int/float> <add/sub/mul/div> <operand\_1> <operand\_2>}

        \item Configure the \texttt{my\_exe} Python script, editing the path to the .JAR file, and then run it with the four arguments, which will automate the above process:\\
        \texttt{<python3/python> my\_exe <int/float> <add/sub/mul/div> <operand\_1> <operand\_2>}\\
        or\\
        \texttt{./my\_exe <int/float> <add/sub/mul/div> <operand\_1> <operand\_2>}

        \item Example Input:\\
        \texttt{./my\_exe float sub 840196454.51725 712586963.70283}\\
        Output:\\
        \texttt{127609490.81442}
    \end{itemize}
\end{itemize}

\section{Conclusion}

\subsection{Key Learnings}

\begin{itemize}
    \item Handling large numbers as Strings and performing arithmetic operations on them.
    \item Object Oriented Programming with Java.
    \item Logic of basic arithmetic operations digit-wise.
    \item Build systems in Java.
    \item How to use .JAR packages in files.
    \item Using Python to build execution scripts.
\end{itemize}

\subsection{Areas of Improvement}

\begin{itemize}
    \item A lot of code is reused within the same class for different cases. For example, trimming zeroes for integers and floats in the \texttt{AFloat} class has the same core logic but slightly different implementation.
    \item Right now I am not handling the operations universally, i.e., I am checking for the positivity of the operands and making cases for each operation.
\end{itemize}


\end{document}